\documentclass[12pt]{article}
\usepackage[paper=letterpaper,margin=2cm]{geometry}
\usepackage{amsmath}
\usepackage{amssymb}
\usepackage{amsfonts}
\usepackage{newtxtext, newtxmath}
\usepackage{enumitem}
\usepackage{titling}
\usepackage[colorlinks=true]{hyperref}

\setlength{\droptitle}{-6em}

\title{Distributionally Robust Risk Parity Portfolios with Wasserstein Distance}
\author{Mohammad Rafay Kalim }
\date{\vspace{-5ex}}

\begin{document}
\maketitle

\section{Project Description}
Many portfolio optimization techniques, such as Mean Variance Optimization (MVO), focus on the tradeoff between risk and reward \cite{markowitz1952}. This requires the estimation of $\mu$, the reward measure, and $\Sigma$, the risk measure.  Contrary to MVO, Risk Parity (RP) portfolios have the objective of ensuring an equal risk contribution from each asset, and thus they traditionally do not require the estimation of $\mu$ \cite{qian2011risk}.  However, the estimation of $\Sigma$ poses the risk of estimation errors making the output of RP sensitive to input data. Recent advances in Distributionally Robust Optimization have led to Distributionally Robust Risk Parity (DRRP), which introduces an uncertainty set around the probability distribution of historical observations used as inputs to RP \cite{costa2020robust}.  However, this thesis will introduce a new version of DRRP, called DRRP-W (for Wasserstein) which assumes a uniform probability distribution for our observed historical returns, but introduces an uncertainty set around the return observations themselves using the Wasserstein Distance.  Computational experiments will be run to compare this new model against DRRP, Robust MVO and Robust Risk Parity. Further, this thesis will propose an end-to-end optimization approach following \cite{butler2023integrating} to optimize the choice of $\delta$, the size of the uncertainty set, which maximizes the out-of-sample sharpe ratio.

\section{Existing Literature}
Blanchet, Chen and Zhou (2018) explore distributionally robust mean-variance portfolios using the Wasserstein distance \cite{blanchet2022distributionally}. They are able to find a convex problem which protects portfolios against varations in $\mu$.  Their paper provides a good structure that I can follow to extend the idea to Risk Parity portfolios.  They also provide an exposition on the critical choice of $\delta$, which is based in 3 assumptions on the distribution of the underlying data.  Butler and Kwon (2023) propose a novel end-to-end optimization technique for various popular portfolio optimization strategies \cite{butler2023integrating}. The goal of this paper is to find an alternative to the common `Predict, then optimize' workflow followed in this field, where the `Predict' component is studied through factor models, and the `optimize' component is studied through MVO, RP, etc. They integrate linear regression OLS optimization into MVO, to get the best regression model that maximizes the out of sample sharpe ratio.  I would like to follow the steps outlined in this paper to guide the choice of $\delta$.

\section{Objectives, Methods and Significance}
In this paper, I would like to propose two new ideas to the field of Distributionally Robust Risk Parity Optimization.  First, using the Wasserstein distance to make RP portfolios robust against variation in historical return data. Second, using end-to-end optimization in determining the size of the uncertainty set. Finally, I will perform various computational studies to determine the impact of these two ideas and compare them against existing portfolio optimization techniques. To the best of my knowledge, these ideas are novel in this field and can present computational and performance advantages over existing techniques.

\newpage

\bibliographystyle{ieeetr}  

\bibliography{citations}


\end{document}
